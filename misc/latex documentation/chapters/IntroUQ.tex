\chapter{Forward Propagation of Uncertainty}
\label{chapter:Forward Propagation of Uncertainty}
\begin{itemize}
\item Brief summary of what the chapter will contain
\end{itemize}
\section{Theoretical Background}
\label{sec:Theoretical Background}
\begin{itemize}
\item Explain what is forward propagation of uncertainty
\item What is assumed to be known
\item ..
\end{itemize}
\section{Probabilistic Modelling}
\label{sec:Probabilistic Modelling}
\begin{itemize}
\item Introduce a generic probabilistic space, where the stochastic modelling is made
\item Reminder of what are univariate and multivariate random variables
\item Important: we work only with iid random vars; plus what this implies
\item Formulas for expectation and variance for one dim and multi dim rand vars
\item ...
\end{itemize}	
\section{Overview of Sampling Methods}
\label{sec:Overview of Sampling Methods}
\begin{itemize}
\item Brief reminder of what sampling methods are and specify some examples
\item Explain polynomial and generalized polynomial chaos methodology (from Wiener until Xiu and collaborators)
\item Why to use gPC (e.g. because of the advancements in HPC, etc)
\item Literature review
\item ...
\end{itemize}	
\section{Monte Carlo Sampling}
\label{sec:Monte Carlo Sampling}
\begin{itemize}
\item Overview of Monte Carlo methodology in the UQ context
\item ...
\end{itemize}	
\subsection{One-Dimensional Case}
\label{subsec:One-Dimensional Case}
\begin{itemize}
\item Describe the algorithm in a one-dimensional stochastic setting
\item ...
\end{itemize}	
\subsection{Multi-Dimensional Case}
\label{subsec:Multi-Dimensional Case}
\begin{itemize}
\item Describe the algorithm in a multi-dimensional stochastic setting
\item ...
\end{itemize}
\subsection{Convergence Analysis and Properties}
\label{subsec:Convergence Analysis and Properties}
\begin{itemize}
\item Properties: e.g. Convergence for generic Monte Carlo sampling + ways to improve it (e.g. quasi Monte Carlo), independence of dimension, distribution, parallelization, numerical errors, etc
\item Pros and Cons; Motivate Collocations
\end{itemize}	
\section{Stochastic Collocations}
\label{sec:Stochastic Collocations}
\begin{itemize}
\item Briefly explain what collocations are
\item Specify that MCS is just a particular case of collocations, using the delta function as stochastic basis
\item ...
\end{itemize}
\subsection{Pseudo-spectral Approach}
\label{subsec:Gaussian Quadrature}
\begin{itemize}
\item Describe what does the pseudo-spectral approach look like and that is based on quadrature
\item Explain Gauss based quadrature: Gauss-Hermite and Gauss-Legendre
\item Explain how to compute the statistics with this approach
\item Explain the restrictions imposed to the mean and standard deviation of the random variable $(\sqrt{2}*node*stddev + mean > 0 => mean/stddev > - \sqrt{2}*node!)$
\item ...
\end{itemize}	
\subsection{One-Dimensional Case}
\label{subsec:One-Dimensional Case}
\begin{itemize}
\item Describe the algorithm in a one-dimensional stochastic setting
\item ..
\end{itemize}	
\subsection{Multi-Dimensional Case}
\label{subsec:Multi-Dimensional Case}
\begin{itemize}
\item Describe the algorithm in a multi-dimensional stochastic setting
\item Curse of dimensionality with brute force approach
\item ..
\end{itemize}
\subsection{Convergence Analysis and Properties}
\label{subsec:Convergence Analysis and Properties}
\begin{itemize}
\item Properties: convergence (exponential also outside the Askey scheme!!), parallelization, non-intrusive, dependence on distribution and dim, numerical errors, etc
\item Pros and Cons of the method 
\item ...
\end{itemize}	
\section{Sparse Grids Stochastic Collocations}
\label{sec:Sparse Grids Stochastic Collocations}
\begin{itemize}
\item Brief overview of SG-SCS
\item A way to (partially) break the curse of dimensionality
\item ...
\end{itemize}
\subsection{Overview of Sparse Grids}
\label{subsec:Overview of Sparse Grids}
\begin{itemize}
\item Overview of sparse grid methodology
\item Quadrature on sparse grids
\item ...
\end{itemize}	
\subsection{Sparse Grids Toolbox SG++}
\label{subsec:Sparse Grids Toolbox SG++}
\begin{itemize}
\item Brief overview of SG++ and its capabilities
\item Explain how quadrature works within SG++
\item Improvements gained from using SG++ (no of collocation points, memory, convergence, etc)
\item ...
\end{itemize}
\section{Probability Density Function Estimation}
\label{sec:Probability Density Function Estimation}
\subsection{General Description}
\label{subsec:General Description}
\begin{itemize}
\item Overview of what density estimation is and why we use it here
\item Some methods: brute force, KDE, etc
\item ...
\end{itemize}
\subsection{Kernel Density Estimation}
\label{subsec:Kernel Density Estimation}
\begin{itemize}
\item Explain what KDE is
\item Examples of Kernels
\item Implementation using Python scikit toolbox
\item ...
\end{itemize}